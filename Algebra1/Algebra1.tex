\documentclass[a4paper,11pt,NoNotes,GeneralMath]{stdmdoc}

\newcommand{\Char}{\text{Char }}
\newcommand{\sgr}{\sqsubseteq}
\newcommand{\nrm}{\lhd}
\newcommand{\Gal}{\text{Gal }}

\begin{document}
	\title{Domande degli Orali di Algebra 1}
	\autodate

    \section*{Preappello di Gennaio 2016}
    \begin{itemize}
        \item Restiamo negli anelli commutativi unitari. Ce ne sono che hanno un numero finito di ideali massimali? Fai un po' di esempi. Posso trovare un anello che abbia $m$ ideali massimali $\forall m \in \bbN$?
        \item Supponiamo ora di avere $A$ un UFD e supponiamo che $\mid A \mid = + \infty$ e supponiamo che $A^{*}$ sia finito. Dimostra che allora $A$ contiene infiniti elementi primi. (Poi ci accontentiamo anche di $\Char A = 0$ ma ci assicura che si fa anche in caratteristica $p$)
        \item Conosci un gruppo finito che abbia esattamente due sottogruppi di indice due? (Suggerimento: I commutatori) Quanti possono essere i sottogruppi di indice due di un gruppo finito, ovvero quali numeri si realizzano?
        \item Prendo i polinomi a coefficienti interi in infinite variabili $\bbZ [(x_\lambda)_{\lambda \in \Lambda}]$. Questo è un UFD? \\
        Ora al posto di infinite variabili mettiamo infiniti esponenti (razionali) $B = \cup_{n \ge 0} \bbZ [ x^{\frac{1}{n}} ]$. $B$ è un UFD? (Suggerimento: ACCP non soddisfatta)
        \item $G$ gruppo che non ha sottogruppi normali di indice finito. Sia $N \nrm G$ e $N$ di ordine finito. Mostrare che $N \sgr Z(G)$ (Cioè $N$ è sottogruppo del centro) (Suggerimento: Azioni furbe?)
        \item Consideriamo $\bbQ \subseteq K_1, \ldots, K_n$ estensioni distinte dei razionali tutte di grado tre su $\bbQ$. Se $n = 2$ quali sono i possibili gradi del composto $K_1K_2$? Voglio ora determinare il minimo $n$ tale che, per ogni scelta di $K_1, \ldots, K_n$ si abbia $9 \mid [K_1 \ldots K_n : \bbQ ]$
        \item Parliamo di $p$-gruppi. $\mid G \mid = p^3$, $G$ non abeliano. Che dimensione ha il centro? Quante sono le classi di coniugio?
        \item Enuncia e dimostra il teorema di Cayley. $\mid G \mid = n$. Supponiamo ora $n = 2d$ con $d$ dispari. Mostrare che $G$ possiede un sottogruppo di ordine $d$. Supponiamo ora che $d$ sia squarefree, quindi $\mid G \mid = 2 p_1 \cdot \ldots \cdot p_k$ con i $p_i$ tutti distinti. Quanti gruppi $G$ di quest'ordine ci sono? (Quanti sono se $H \sgr G, \mid H \mid = d$, $H$ è ciclico?)
        \item Prendiamo $p=13$ un primo e consideriamo $\bbQ(\zeta_p)$. Quante sottoestensioni su $\bbQ$ di grado 2 ci sono? Dato un sottogruppo di indice $m$ di $C_p^{*} \equiv \Gal(\frac{\bbQ(\zeta_p)}{\bbQ})$ vorrei trovare un sottocampo di grado $m$ su $\bbQ$, espresso come $F = \bbQ(\alpha)$. Dimostrare che è proprio lui ciò che cerchiamo.
        \item Prendo $p, q$ primi distinti e vorrei dire che ogni gruppo di ordine $p^2 q$ si scrive come prodotto semidiretto
        \item Prendo $\bbZ$ e vorrei trovare due sottoinsiemi moltiplicativamente chiusi distinti $S_1, S_2$ di $\bbZ$ tali che $S_1^{-1} \bbZ \equiv S_2^{-1} \bbZ$
        \item Prendo un ideale in $A = \bbZ [x]$. è vero che se non è massimale possono non bastare solo due generatori (al contrario di come invece avviene per quelli massimali)? (Riportiamo la soluzione integrale: Vero, notiamo che per $I = (4, 2x, x^2) = ((2, x))^2 $ e $(2, x) = M$ è un ideale massimale. Quindi $\frac{M}{M^2}$ è uno spazio vettoriale di dimensione $2$ su $\frac{A}{M}$, che è un campo. Se prendo invece $\frac{M^2}{M^3}$ e controllo che ha dimensione 3 concludo, perché il quoziente ha tre generatori quindi anche $M^2$ deve averne almeno tre.)
		\item Sia $G$ un $p$-gruppo. $\mid G \mid = p^n$ e supponiamo che sia abeliano. Mostra che il numero di sottogruppi di ordine $p$ è congruo a $1$ modulo $p$. Vedere che ciò avviene anche per $G$ non abeliano e per sottogruppi di ogni ordine.
		\item Consideriamo l'estensione ciclotomica $K = \bbQ (\zeta_p)$, con $p$ primo diverso da due. Esso avrà quindi un'unica sottoestensione (su $\bbQ$) di grado $2$ ed un'unica di grado $\frac{p-1}{2}$. Un'estensione quadratica di $\bbQ$ si può sempre esprimere come $\bbQ(\sqrt{a})$. Al variare di $p$ stabilire se $a$ è positivo o negativo.
		\item Com'è fatto un $2$-Sylow di $S7$? (Suggerimento piccolissimo: $7$ in base due si scrive come $1*2^2 + 1 *2^1 + 1*2^0$)
	\end{itemize}
	
	\section*{Sessione di Fine Gennaio 2016}
	\begin{itemize}
		\item Trova un esempio di anello commutativo unitario $A$ nel quale esiste una catena ascendente infinita e non stazionaria di ideali $I_1 \subsetneq I_2 \subsetneq \ldots$
		\item $G$ gruppo finito e $H \sgr G$ un sottogruppo che interseca ogni classe di coniugio. Ovvero $\forall x \in G \exists g \in G \quad \tc gxg^{-1} \in H$. Dimostrare allora che $H = G$. (Suggerimenti: Considerare l'azione di coniugio di $G$ sui sottogruppi coniugati di $H$ e dimostrare che $H$ è normale)
		\item Costruisci un'estensione di Galois con gruppo di Galois $D_5$ (puoi scegliere tu i campi)
		\item Come sono fatti i $2$-Sylow ed i $3$-Sylow di $S_9$. Che gruppi sono?
		\item L'unione di tre sottogruppi propri (tali che nessuno sia contenuto in qualcun'altro) può essere un sottogruppo? Se $G = C_p \times C_p$ si può fare (esistono tre sottogruppi tali che)? Qual è il minimo numero di sottogruppi propri tali che la loro unione sia tutto $C_p \times C_p$?
		\item $A$ dominio euclideo. $B \subset A$ è un sottoanello (unitario). è vero o no che $B$ è a sua volta un dominio euclideo? (Con una qualunque funzione grado)
		\item Prendiamo il campo di spezzamento di un polinomio di terzo grado su $\bbQ$. Quali radici dell'unità può contenere? (Ovvero $\zeta_n$ per quali $n$). E se prendiamo un polinomio di grado quattro? (Suggerimento: usando corrispondenza di Galois deve essere che il gruppo $\text{Gal }(\bbQ(\zeta_n) / \bbQ)$ si ottiene come quoziente di $S_4$)
		\item Supponiamo che $G$ sia generato da $n$ elementi $\langle x_1, x_2, \ldots, x_n \rangle$. $H \sgr G$. Esistono per forza $\ge n$ elementi che generano $H$? E se $G$ è abeliano finito?
		\item Ci sono anelli c.u. $A$ tali che $(A^{*}, \cdot) \cong (\bbZ, +)$ come gruppo? (Suggerimento: $(\bbZ, +)$ ha solo elementi di ordine infinito. Può essere che $\bbZ_m$ si immerga in $A$ (come sottoanello)? Che ordine ha $\{ -1 \}$? Se esiste deve avere $\bbZ_2$ come sottoanello fondamentale. A questo punto posso considerare $S = \{ 1, x, x^2, \ldots \}$ e $B = \bbF_2[x]$ allora $A = S^{-1}B$ è un anello cercato.
		\item Classificare i gruppi di ordine $3 \cdot 5 \cdot 7$. Chi sono i sottogruppi normali di $\bbZ_{35} \rtimes_\phi \bbZ_3$?
    \end{itemize}
    
\end{document}
