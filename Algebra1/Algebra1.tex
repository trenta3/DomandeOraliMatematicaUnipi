\documentclass[a4paper,11pt,NoNotes,GeneralMath]{stdmdoc}

\newcommand{\Char}{\text{Char }}
\newcommand{\sgr}{\sqsubseteq}
\newcommand{\nrm}{\lhd}
\newcommand{\Gal}{\text{Gal }}

\begin{document}
	\title{Domande degli Orali di Algebra 1}
	\autodate

    \section*{Preappello di Gennaio 2016}
    \begin{itemize}
        \item Restiamo negli anelli commutativi unitari. Ce ne sono che hanno un numero finito di ideali massimali? Fai un po' di esempi. Posso trovare un anello che abbia $m$ ideali massimali $\forall m \in \bbN$?
        \item Supponiamo ora di avere $A$ un UFD e supponiamo che $\mid A \mid = + \infty$ e supponiamo che $A^{*}$ sia finito. Dimostra che allora $A$ contiene infiniti elementi primi. (Poi ci accontentiamo anche di $\Char A = 0$ ma ci assicura che si fa anche in caratteristica $p$)
        \item Conosci un gruppo finito che abbia esattamente due sottogruppi di indice due? (Suggerimento: I commutatori) Quanti possono essere i sottogruppi di indice due di un gruppo finito, ovvero quali numeri si realizzano?
        \item Prendo i polinomi a coefficienti interi in infinite variabili $\bbZ [(x_\lambda)_{\lambda \in \Lambda}]$. Questo è un UFD? \\
        Ora al posto di infinite variabili mettiamo infiniti esponenti (razionali) $B = \cup_{n \ge 0} \bbZ [ x^{\frac{1}{n}} ]$. $B$ è un UFD? (Suggerimento: ACCP non soddisfatta)
        \item $G$ gruppo che non ha sottogruppi normali di indice finito. Sia $N \nrm G$ e $N$ di ordine finito. Mostrare che $N \sgr Z(G)$ (Cioè $N$ è sottogruppo del centro) (Suggerimento: Azioni furbe?)
        \item Consideriamo $\bbQ \subseteq K_1, \ldots, K_n$ estensioni distinte dei razionali tutte di grado tre su $\bbQ$. Se $n = 2$ quali sono i possibili gradi del composto $K_1K_2$? Voglio ora determinare il minimo $n$ tale che, per ogni scelta di $K_1, \ldots, K_n$ si abbia $9 \mid [K_1 \ldots K_n : \bbQ ]$
        \item Parliamo di $p$-gruppi. $\mid G \mid = p^3$, $G$ non abeliano. Che dimensione ha il centro? Quante sono le classi di coniugio?
        \item Enuncia e dimostra il teorema di Cayley. $\mid G \mid = n$. Supponiamo ora $n = 2d$ con $d$ dispari. Mostrare che $G$ possiede un sottogruppo di ordine $d$. Supponiamo ora che $d$ sia squarefree, quindi $\mid G \mid = 2 p_1 \cdot \ldots \cdot p_k$ con i $p_i$ tutti distinti. Quanti gruppi $G$ di quest'ordine ci sono? (Quanti sono se $H \sgr G, \mid H \mid = d$, $H$ è ciclico?) (Mostrare poi che se $\mid G \mid = 2d$ allora $H \sgr G$ di ordine $d$ è abeliano)
        \item Prendiamo $p=13$ un primo e consideriamo $\bbQ(\zeta_p)$. Quante sottoestensioni su $\bbQ$ di grado 2 ci sono? Dato un sottogruppo di indice $m$ di $C_p^{*} \equiv \Gal(\frac{\bbQ(\zeta_p)}{\bbQ})$ vorrei trovare un sottocampo di grado $m$ su $\bbQ$, espresso come $F = \bbQ(\alpha)$. Dimostrare che è proprio lui ciò che cerchiamo.
        \item Prendo $p, q$ primi distinti e vorrei dire che ogni gruppo di ordine $p^2 q$ si scrive come prodotto semidiretto
        \item Prendo $\bbZ$ e vorrei trovare due sottoinsiemi moltiplicativamente chiusi distinti $S_1, S_2$ di $\bbZ$ tali che $S_1^{-1} \bbZ \equiv S_2^{-1} \bbZ$
    \end{itemize}
    
\end{document}
