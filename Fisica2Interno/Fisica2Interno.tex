\documentclass[a4paper,NoNotes,GeneralMath]{stdmdoc}
\usepackage{pgf}
\usepackage{tikz}
\usetikzlibrary{arrows,automata}

\begin{document}
	\title{Domande Fisica 2 Interno - 2015/16}
	
	\section*{Colloqui}
	Non ho scritto le domande relative ad esercizi del compito, che comunque ci sono state
	\begin{itemize}
		\item Deduci la forma delle quattro equazioni di Maxwell
		\item Parla dei legami fra $\vec{B}, \vec{E}$ e le onde
		\item Dipoli oscillanti. Perché è assurdo che una carica in moto rettilineo uniforme irraggi? (Come va asintoticamente l'irraggiamento?)
		\item Filo infinito lungo l'asse $\hat{x}$ con corrente costante $I$ e densità lineare di carica costante $\lambda$. Trovare i campi \\
			Sotto quali condizioni su $\lambda$ e $I$ esiste un boost che elimina uno dei due campi? Trovare il $\beta$ di un boost che elimina il campo magnetico (si fa con le sorgenti e non con i campi)
		\item Armoniche sferiche (scrittura generale), sviluppo in multipoli e formula di addizione
		\item Potenziali ritardati
		\item L'energia elettrostatica: formula con $u = \frac{E^2}{8\pi }$ che è sempre positiva, e formula dell'energia potenziale tra due cariche $U = \frac{q_1 q_1}{d}$, che può essere anche negativa. Motivo della differenza
		\item Definizione del tensore di Maxwell, bilancio della quantità di moto (niente conti, solo enunciati)
		\item Esercizio con una sbarretta che cade tra due binari metallici. I binari sono in circuito con un condensatore. C'è un campo magnetico costante perpendicolare al piano su cui si muove la sbarra.
	\end{itemize}

\end{document}
