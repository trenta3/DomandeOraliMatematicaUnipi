\documentclass[a4paper,NoNotes,GeneralMath]{stdmdoc}
\usepackage{pgf}
\usepackage{tikz}
\usetikzlibrary{arrows,automata}

\begin{document}
	\title{Domande Fisica 2 Interno - 2015/16}
	
	\section*{Colloqui}
	Non ho scritto le domande relative ad esercizi del compito, che comunque ci sono state
	\begin{itemize}
		\item Deduci la forma delle quattro equazioni di Maxwell
		\item Parla dei legami fra $\vec{B}, \vec{E}$ e le onde
		\item Dipoli oscillanti. Perché è assurdo che una carica in moto rettilineo uniforme irraggi? (Come va asintoticamente l'irraggiamento?)
		\item Filo infinito lungo l'asse $\hat{x}$ con corrente costante $I$ e densità lineare di carica costante $\lambda$. Trovare i campi \\
			Sotto quali condizioni su $\lambda$ e $I$ esiste un boost che elimina uno dei due campi? Trovare il $\beta$ di un boost che elimina il campo magnetico (si fa con le sorgenti e non con i campi)
		\item Armoniche sferiche (scrittura generale), sviluppo in multipoli e formula di addizione
		\item Potenziali ritardati
		\item L'energia elettrostatica: formula con $u = \frac{E^2}{8\pi }$ che è sempre positiva, e formula dell'energia potenziale tra due cariche $U = \frac{q_1 q_1}{d}$, che può essere anche negativa. Motivo della differenza
		\item Definizione del tensore di Maxwell, bilancio della quantità di moto (niente conti, solo enunciati)
		\item Esercizio con una sbarretta che cade tra due binari metallici. I binari sono in circuito con un condensatore. C'è un campo magnetico costante perpendicolare al piano su cui si muove la sbarra.
		\item Soluzione più rapida dell'esercizio quattro che non sia farlo a conti bruti (Considerare il quadripotenziale e vedere se è di tipo tempo / spazio)
		\item Equazione d'onda, forma generale delle soluzioni e dimostrare che vale per una corda vibrante uniforme a tensione costante per piccoli spostamenti
		\item Far vedere una soluzione delle equazioni di Maxwell (Hint: Bastano le onde piane)
		\item "Lungo il suo asse, un solenoide tende ad allungarsi o accorciarsi? Soluzione più semplice: chiudere circuitalmente il solenoide e vedere che in questo modo anche variando la lunghezza il flusso magnetico è costante; imponendo che sia costante si calcola l’energia in funzione della lunghezza e viene proporzionale alla lunghezza, quindi tende ad accorciarsi."
		\item Differenza tra flusso variato e flusso tagliato
		\item Dimostra che Biot-Savart implica Maxwell
		\item Quali sono gli invarianti relativistici dei campi?
		\item Tensore elettromagnetico ed invarianti che si ottengono da lui e dal suo duale (e che cosa possiamo dedurre dai valori che assumono)
		\item Cariche su un anello dentro un solenoide (coassiale all'anello). All'inizio l'anello è fermo e nel solenoide c'è corrente, poi lentamente faccio calare la corrente fino a 0. A questo punto cos'è successo? Risposta: il campo elettrico dato dalla variazione di B mi ha messo in moto le cariche che girano per l'anello... Domanda vera: ma all'inizio erano ferme, quindi da dove è venuto il momento angolare?
	\end{itemize}

\end{document}
