\documentclass[a4paper,NoNotes,GeneralMath]{stdmdoc}
\usepackage{pgf}
\usepackage{tikz}
\usetikzlibrary{arrows,automata}

\begin{document}
\title{Domande Geometria Differenziale Interno - 2016/2017}

\section*{Domande dell'esame finale}
\begin{itemize}
\item Data una $k$-distribuzione $\Delta$ di base locale $\{X_1, \ldots, X_k\}$ tali che $[X_i, X_j] \in \Delta$, mostrare che esiste una base locale di $\Delta$ tale che la nuova base abbia commutatori nulli
\item Dimostrare la prima identità di Bianchi
\item Dare un controesempio di distribuzione non integrabile
\item Domanda finale: Quali sono le applicazioni delle varietà integrabili nella vita quotidiana? Il problema del Parcheggio
\item Dimostrare il Teorema di Froebenius
\item Dimostrare l'esistenza e l'unicità della connessione di Levi-Civita
\item Definizione della Derivata di Lie e relative formule
\item Dimostrazione della formula magica di Cartan
\item Definizione di distribuzione e di distribuzione integrabile
\item Definizione del commutatore di due campi. Mostrare che è un campo vettoriale anch'esso. Mostrare che dati $X, Y$ campi (derivazioni) $XY$ non è una derivazione
\item Dimostrare $\nabla$ è compatibile $\sse$ $\nabla_X (g(Y, Z)) = g(\nabla_X Y, Z) + g(Y, \nabla_X Z)$ per ogni campo vettoriale $X, Y, Z$
\item Definizione di forma differenziale
\item Definizione dell'operatore di derivata esterna, sia in carta che in astratto, e verifica di alcune sue proprietà
\item Definizione di curvatura e sue proprietà
\item Dati due campi vettoriali $X, Y$ e relativi flussi $\theta_X, \theta_Y$ si ha che $[X, Y] = 0 \sse \theta_X^{t} \circ \theta_Y^{s} = \theta_Y^{s} \circ \theta_X^{t}$
\item Definizione di Fibrato vettoriale
\item Definizione di Moving Frames
\item Formule per le forme di connessione e formula di struttura
\item Formula di controvarianza
\item Definizione dei simboli di Christoffel
\item Dimostrazione di esistenza ed unicità delle geodetiche
\item Definizione di Connessione Affine
\item Definizione di Curvatura Sezionale ed invarianza per cambio di base
\item Esistenza ed unicità della derivata esterna
\item Equazione di Struttura, esistenza ed unicità delle omega
\item Cambio di coordinate di un tensore generico
\item Dimostra che ogni varietà differenziabile ammette una metrica Riemanniana
\end{itemize}
        
\end{document}
