\documentclass[a4paper,NoNotes,GeneralMath]{stdmdoc}
\usepackage{pgf}
\usepackage{tikz}
\usetikzlibrary{arrows,automata}
\newcommand{\SU}{\text{SU }}
\newcommand{\SO}{\text{SO }}
\newcommand{\U}{\text{U }}

\begin{document}
	\title{Domande Rappresentazioni Interno - 2015/16}

	\section*{Colloqui}
	\begin{itemize}
		\item Rappresentazioni irriducibili di $\SU(2)$
		\item Quali rappresentazioni irriducibili di $S^1$ si possono estendere a rappresentazione irriducibili di $\SU(2)$
		\item È vero che tutte le rappresentazioni irriducibili di un sottogruppo $H$ di $G$ si possono estendere a rappresentazioni di $G$?
		\item Se in $\rho \otimes \sigma$ (con $\rho$ e $\sigma$ irriducibili) ho una sottorappresentazione di grado $1$, cosa posso dire?
		\item Omomorfismo tra $\SU(2)$ e $\SO(3)$ (mostrare che è surgettivo e che ha $\Ker = \pm \Id$ )
		\item Controesempio a Schur $2$ sui reali
		\item Quali sono le rappresentazioni reali irriducibili di $S^1$?
		\item Quali delle rappresentazioni di $\U(2)$ troviamo con la stessa costruzione con cui abbiamo trovato quelle di $\SU(2)$? (Facendole agire su $\bbC^2$)
		\item ({\bf Gruppo di Eisenstein}) Matrici invertibili $2\times 2$ a coefficienti in $\bbF_p$ triangolari superiori tali che $a_{22} = 1$ (ovvero le affinità di $\bbF_p$) \\
			Trovarne le classi di coniugio (e cardinalità) e le rappresentazioni irriducibili su $\bbC$
		\item $G$ finito con $\rho$ irriducibile e fedele. Allora $Z(G)$ è ciclico
		\item Sia $V_m$ una rappresentazione irriducibile di $\SU(2)$. Come si scompone in irriducibili $V_m^*$?
		\item Cosa può accadere ad una rappresentazione complessa irriducibile dopo che la realifico?
		\item Quali delle $V_m$ (sempre per $\SU(2)$) sono reali? Ovvero trova una forma bilineare su queste e dì se è simmetrica o alternante. (Viene diviso in base ai casi $m$ pari / dispari)
		\item Prendi un'azione di $G$ su $X$ e la corrispondente rappresentazione per permutazione $V$. Dimostra che se l'azione di $G$ su $X\times X$ è doppiamente transitiva allora la rappresentazione ortogonale al sottospazio $\span{e_1 + e_2 + \ldots + e_n}$ è irriducibile
		\item Dando per buono che le uniche algebre di divisione finito dimensionali su $\bbR$ sono $\bbR, \bbC, \bbH$ (quaternioni) dimostra che se $\rho$ è quaternionica (ovvero ammette una forma quadratica alternante) allora gli endomorfismi di rappresentazioni della sua realificata sono isomorfi a $\bbH$
		\item Discussione libera su realificazione, complessificazione
		\item Quali rappresentazioni irriducibili di $\SU(2)$ sono complessificate di rappresentazione reali irriducibili?
	\end{itemize}
\end{document}
