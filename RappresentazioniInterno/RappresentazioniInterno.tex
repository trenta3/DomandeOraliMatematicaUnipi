\documentclass[a4paper,NoNotes,GeneralMath]{stdmdoc}
\usepackage{pgf}
\usepackage{tikz}
\usetikzlibrary{arrows,automata}
\newcommand{\SU}{\text{SU }}
\newcommand{\SO}{\text{SO }}
\newcommand{\U}{\text{U }}
\newcommand{\Hom}{\text{Hom }}

\begin{document}
	\title{Domande Rappresentazioni Interno - 2015/16}

	\section*{Colloqui}
	\begin{itemize}
		\item Rappresentazioni irriducibili di $\SU(2)$
		\item Quali rappresentazioni irriducibili di $S^1$ si possono estendere a rappresentazione irriducibili di $\SU(2)$ ?
		\item È vero che tutte le rappresentazioni irriducibili di un sottogruppo $H$ di $G$ si possono estendere a rappresentazioni di $G$?
		\item Se in $\rho \otimes \sigma$ (con $\rho$ e $\sigma$ irriducibili) ho una sottorappresentazione di grado $1$, cosa posso dire?
		\item Omomorfismo tra $\SU(2)$ e $\SO(3)$ (mostrare che è surgettivo e che ha $\Ker = \pm \Id$ )
		\item Dimostra Schur $2$ per rappresentazioni complesse e dai un controesempio a Schur $2$ sui reali.
		\item Quali sono le rappresentazioni reali irriducibili di $S^1$?
		\item Quali delle rappresentazioni di $\U(2)$ troviamo con la stessa costruzione con cui abbiamo trovato quelle di $\SU(2)$? (Facendole agire sulle potenze simmetriche di $\bbC^2$) [Non sono tutte, ma quali tra queste si trovano]
		\item ({\bf Gruppo di Eisenstein}) Matrici invertibili $2\times 2$ a coefficienti in $\bbF_p$ triangolari superiori tali che $a_{22} = 1$ (ovvero le affinità di $\bbF_p$) \\
			Trovarne le classi di coniugio (e cardinalità) e le rappresentazioni irriducibili su $\bbC$
		\item $G$ finito con $\rho$ irriducibile e fedele. Allora $Z(G)$ è ciclico
		\item Sia $V_m$ una rappresentazione irriducibile di $\SU(2)$. Come si scompone in irriducibili $V_m^*$?
		\item Cosa può accadere ad una rappresentazione complessa irriducibile dopo che la realifico?
		\item Quali delle $V_m$ (sempre per $\SU(2)$) sono reali? Ovvero trova una forma bilineare su queste e dì se è simmetrica o alternante. (Viene diviso in base ai casi $m$ pari / dispari)
		\item Prendi un'azione di $G$ su $X$ e la corrispondente rappresentazione per permutazione $V$. Dimostra che se l'azione di $G$ su $X\times X$ è doppiamente transitiva allora la rappresentazione ortogonale al sottospazio generato da ${e_1 + e_2 + \ldots + e_n}$ è irriducibile
		\item Dando per buono che le uniche algebre di divisione finito dimensionali su $\bbR$ sono $\bbR, \bbC, \bbH$ (quaternioni) dimostra che se $\rho$ è quaternionica (ovvero ammette una forma quadratica alternante) allora gli endomorfismi di rappresentazioni della sua realificata sono isomorfi a $\bbH$
		\item Discussione libera su realificazione, complessificazione
		\item Quali rappresentazioni irriducibili di $\SU(2)$ sono complessificate di rappresentazione reali irriducibili?
		\item È sempre vero che $\Dim \Hom(\sigma, \rho) = \Dim \Hom(\rho, \sigma)$? (Si intende per ogni gruppo qualunque, per ogni due rappresentazioni) (Hint: No, bisogna considerare delle rappresentazioni di $\bbZ^2$)
		\item Le matrici diagonali dentro $\U(2)$ sono un sottogruppo isomorfo a $S^1 \times S^1$. Quali caratteri di rappresentazioni di $S^1 \times S^1$ si possono ottenere restringendo una rappresentazione di $\U(2)$? (In particolare si possono ottenere $\lambda + \mu$, $\lambda\mu$, $\lambda^2 + \mu^2$ dove $\lambda, \mu$ sono i due autovalori che compaiono nella diagonalizzata di una matrice di $\U(2)$
		\item Parla dell'ortogonalità dei caratteri.
		\item Integrazione invariante su $\SU(2)$ con formula esplicita in generale ed in particolare su $S^1$
		\item Dimostrare che le rappresentazioni di $\U(2)$ ottenute come nel secondo esercizio del compitino sono tutte le irriducibili
		\item (In realtà poi ha cambiato domanda) Perché $\frac{1}{2}$ non può comparire nella tavola dei caratteri di gruppi finiti?
		\item Come sono le potenze esterne delle irriducibili di $\SU(2)$?
		\item Considera la rappresentazione di $\SO(2)$ (oppure $\SU(2)$, chi ha avuto la domanda non se lo ricorda) definita da: $A \mapsto \left( \begin{array}{cc} A & 0 \\ 0 & 1 \\ \end{array} \right)$ a valori nelle matrici $3\times 3$. Si chiede di scomporla in irriducibili
		\item Tra tutte le rappresentazioni di $A_n$ ne esiste una di grado minimo (tolta la banale). Sia $\Deg(n)$ tale numero naturale. Mostrare allora che al variare di $n \in \bbN$, $\Deg(n)$ non è limitata.
		\item Se ho $\phi: G \rar G'$ morfismo di gruppi, quando posso sollevarlo ad un morfismo da $H \rtimes G \rar H \rtimes G'$? Usando il fatto che il gruppo dell'esercizio numero tre era $\bbZ_2 \rtimes (\bbZ_3 \times \bbZ_3)$ e che $S_3$ è $\bbZ_2 \rtimes \bbZ_3$, cerca di sollevare le rappresentazioni di $S_3$ e vedi quante distinte ne vengolo (vengono tutte e quattro le non banali)
		\item Come classificheresti le rappresentazioni di $\SU(3)$?
	\end{itemize}
\end{document}
